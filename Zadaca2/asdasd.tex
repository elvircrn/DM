\documentclass{article}
\usepackage[utf8x]{inputenc}
\usepackage[utf8]{amsmath}
\usepackage{fancyhdr}
\usepackage{listings}
\usepackage{amsfonts}
\usepackage{enumitem}

\usepackage{tikz}

% Generated by LaTreeX, <http://www.lautgesetz.com/latreex/>
% To use this tree as a template, click Paste Template and paste the following line.
% {"LFTsep":"4pt","LFTwidth":"15ex","font":"sf","levelsep":"2cm","linewidth":"0.3pt","orient":"D","style":"nonflat","tree":"-0.1 + 0.25\r\n--0.1 + 0.25\r\n---0.1 + 0.15\r\n----0.1 + 0.05\r\n-----x4 \\(\\rightarrow\\) 0000\r\n-----x2$\\rightarrow$ 0001\r\n----x5$\\rightarrow$ 001\r\n---x3$\\rightarrow$ 01\r\n--0.25 + 0.2\r\n---x6$\\rightarrow$ 10\r\n---x1$\\rightarrow$ 11","treesep":"4ex"}

\usepackage{times,varwidth,pst-tree,pst-eps}
\pagestyle{empty}
\psset{showbbox=false,treemode=D,linewidth=0.3pt,treesep=4ex,levelsep=2cm}
\newcommand{\LFTw}[2]{%
\Tr[ref=#1]{\psframebox[linestyle=none,framesep=4pt]{%
\begin{varwidth}{15ex}\center #2\end{varwidth}}}}
\newcommand{\LFTr}[2]{\Tr[ref=#1]{\psframebox[linestyle=none,framesep=4pt]{#2}}}

\def\pstreehooki{\psset{thislevelsep=*0pt}}
\def\pstreehookiii{\psset{thislevelsep=*0pt}}
\def\pstreehookv{\psset{thislevelsep=*0pt}}
\def\pstreehookvii{\psset{thislevelsep=*0pt}}



\usepackage[margin=1in,headheight=13.6pt]{geometry}


\DeclareMathSizes{10}{9}{8}{7}   


\title{DM Zadaća 3}
\author{Elvir Crnčević}
\date{Decembar 2016}

\usepackage{natbib}
\usepackage{graphicx}

\begin{document}

\pagestyle{fancy}
\lhead {Elvir Crnčević \\ Broj indeksa: 17455}
\rhead {DM Zadaća 3 \\ Decembar, 2016}
\maketitle

\newcommand{\overbar}[1]{\mkern 1.5mu\overline{\mkern-1.5mu#1\mkern-1.5mu}\mkern 1.5mu}
\newpage

\section*{Zadatak 1}
Tražene vrijednosti nalazimo prema sljedećim formulama:

\begin{equation*}
\begin{aligned}
p(A_i B_j) &= p (B_j \mid A_i) p(A_i) \\
H(X, Y) &= -\sum_{j = 1}^{4}\sum_{i = 1}^{3}p(A_i B_j) log_2(p(A_i B_j)) = 3.04551 \\
H(Y \mid X) &= -\sum_{j = 1}^{4}\sum_{i = 1}^{3}p(A_i B_i) log_2(p(B_j \mid A_i)) = 1.72765\\
H(X) &= H(X, Y) - H(Y \mid X) = 1.31786 \\
p(B_j) &= \sum_{i = 1}^3p(A_i B_j) \\
H(Y) &= \sum_{j = 1}^4p(B_j) log_2(p(B_j)) = 1.31786 \\
I(X, Y) &= H(Y) - H(Y \mid X) = 0.198652 \\
H(X, Y) &= H(X) + H(Y \mid X) = 3.04551
\end{aligned}
\end{equation*}

pri čemu je \(
X = \{ A_1, A_2, A_3 \}\) i
\(Y = \{ B_1, B_2, B_3, B_4 \}\).

\section*{Zadatak 3}

Dijagram stanja:

\begin{center}
\begin{tikzpicture}[scale=0.2]
\tikzstyle{every node}+=[inner sep=0pt]
\draw [black] (64.2,-7.3) circle (3);
\draw (64.2,-7.3) node {$S_d$};
\draw [black] (13.7,-11.1) circle (3);
\draw (13.7,-11.1) node {$S_a$};
\draw [black] (63,-44.7) circle (3);
\draw (63,-44.7) node {$S_c$};
\draw [black] (13.7,-47.1) circle (3);
\draw (13.7,-47.1) node {$S_b$};
\draw [black] (61.326,-8.159) arc (-74.13392:-97.25957:111.706);
\fill [black] (16.67,-11.52) -- (17.4,-12.12) -- (17.53,-11.12);
\draw (39.5,-12.8) node [below] {$a\mbox{ }/\mbox{ }0.4$};
\draw [black] (10.89,-10.082) arc (277.81409:-10.18591:2.25);
\draw (6.04,-5.75) node [above] {$a\mbox{ }/\mbox{ }0.4$};
\fill [black] (12.8,-8.25) -- (13.19,-7.39) -- (12.19,-7.53);
\draw [black] (66.266,-5.141) arc (163.98311:-124.01689:2.25);
\draw (71.3,-4.21) node [right] {$d\mbox{ }/\mbox{ }0.2$};
\fill [black] (67.17,-7.63) -- (67.8,-8.33) -- (68.08,-7.37);
\draw [black] (61.848,-41.931) arc (-159.45446:-204.22101:41.961);
\fill [black] (61.85,-41.93) -- (62.04,-41.01) -- (61.1,-41.36);
\draw (58.65,-25.85) node [left] {$c\mbox{ }/\mbox{ }0.3$};
\draw [black] (60.12,-45.541) arc (-74.59076:-99.83515:99.588);
\fill [black] (16.65,-47.66) -- (17.35,-48.29) -- (17.52,-47.3);
\draw (38.67,-49.67) node [below] {$b\mbox{ }/\mbox{ }0.45$};
\draw [black] (14.927,-13.837) arc (22.02981:-22.02981:40.692);
\fill [black] (14.93,-13.84) -- (14.76,-14.77) -- (15.69,-14.39);
\draw (18.4,-29.1) node [right] {$a\mbox{ }/\mbox{ }0.5$};
\draw [black] (13.615,-50.087) arc (26.10273:-261.89727:2.25);
\draw (7.03,-53.58) node [below] {$b\mbox{ }/\mbox{ }0.05$};
\fill [black] (11.28,-48.85) -- (10.34,-48.75) -- (10.78,-49.65);
\draw [black] (65.906,-45.398) arc (104.22902:-183.77098:2.25);
\draw (68.74,-50.45) node [right] {$c\mbox{ }/\mbox{ }0.15$};
\fill [black] (64.21,-47.43) -- (63.93,-48.33) -- (64.89,-48.08);
\draw [black] (15.352,-44.596) arc (145.70586:110.77876:97.64);
\fill [black] (61.38,-8.32) -- (60.45,-8.14) -- (60.81,-9.07);
\draw (32.55,-22.43) node [above] {$d\mbox{ }/\mbox{ }0.2$};
\draw [black] (60.101,-43.927) arc (-105.93067:-142.62145:85.784);
\fill [black] (15.48,-13.52) -- (15.57,-14.45) -- (16.36,-13.85);
\draw (31.86,-32.82) node [below] {$a\mbox{ }/\mbox{ }0.05$};
\draw [black] (16.604,-46.348) arc (103.77301:81.80108:114.09);
\fill [black] (60.04,-44.23) -- (59.32,-43.62) -- (59.17,-44.61);
\draw (38.05,-42.54) node [above] {$c\mbox{ }/\mbox{ }0.25$};
\draw [black] (12.746,-44.256) arc (-163.10563:-196.89437:52.153);
\fill [black] (12.75,-44.26) -- (12.99,-43.35) -- (12.04,-43.64);
\draw (10,-29.1) node [left] {$b\mbox{ }/\mbox{ }0.1$};
\draw [black] (16.527,-10.096) arc (108.6119:79.99461:90.745);
\fill [black] (61.25,-6.73) -- (60.55,-6.1) -- (60.38,-7.08);
\draw (38.34,-4.9) node [above] {$d\mbox{ }/\mbox{ }0.2$};
\draw [black] (65.58,-9.963) arc (24.97645:-28.65192:35.673);
\fill [black] (65.58,-9.96) -- (65.46,-10.9) -- (66.37,-10.48);
\draw (69.45,-26.18) node [right] {$d\mbox{ }/\mbox{ }0.35$};
\draw [black] (16.536,-12.077) arc (70.18955:41.25833:107.833);
\fill [black] (61.05,-42.42) -- (60.9,-41.49) -- (60.15,-42.15);
\draw (43.69,-23.92) node [above] {$c\mbox{ }/\mbox{ }0.3$};
\draw [black] (62.739,-9.92) arc (-30.21223:-73.30315:80.003);
\fill [black] (16.59,-46.29) -- (17.5,-46.54) -- (17.21,-45.58);
\draw (46.16,-32.99) node [below] {$b\mbox{ }/\mbox{ }0.1$};
\end{tikzpicture}
\end{center}

\newpage

Potrebno je rješiti sistem,

\begin{equation*}
    \begin{aligned}
        0 &= -0.6 p(S_a) + 0.5 p(S_a) +  0.05 p(S_a) +  0.4 p(S_a) \\
0 &= 0.1 p(S_b)  -0.95 p(S_b) +  0.45 p(S_b) +  0.1 p(S_b) \\
0 &= 0.3 p(S_c) +  0.25 p(S_c)  -0.85 p(S_c) +  0.3 p(S_c) \\
1 &= p(S_a) +   p(S_b) +  p(S_c) +  p(S_d) \\
    \end{aligned}
\end{equation*}

što ćemo učiniti pomoću QR faktorizacije, pa dobijamo
\begin{equation*}
    \begin{aligned}
p(S_a) &= 0.329388 \\
p(S_b) &= 0.179592 \\
p(S_c) &= 0.253061 \\
p(S_d) &= 0.237959 
    \end{aligned}
\end{equation*}

Dalje računamo \(H(S_i)\) prema formuli \(H(S_i) = -\sum_{j = 1}^{4}p(x_j / S_i)log_2(x_j \mid S_i) \) i dobijamo

\begin{equation*}
    \begin{aligned}
H(S_1) &= 1.38205\\
H(S_2) &= 1.2161\\
H(S_3) &= 1.14504\\
H(S_4) &= 1.38205
    \end{aligned}
\end{equation*}

Entropija izvora je \(H (X \mid X^\infty) = \prod_{i = 1}^4p(S_i)H(S_i) = 1.29227\), a redundansa izvora je 
\begin{equation*}
    \Re = \frac{log_2(4) - H(X \mid X^\infty)}{log_2(4)} = 0.353865
\end{equation*}
Vjerovatnoću pojave sekvence \( adbbadb \) računamo po formuli
\begin{equation*}
    p(adbbadb) = p(S_1)p(d\mid a)p(b \mid d)p(b \mid b)p(d \mid b)p(p b \mid d) = 0.000164694
\end{equation*}
dok vjerovatnoću pojave sekvence dužine 4 računamo pomoću
\begin{equation*}
    \begin{aligned}
    H(X^1) &= -\sum_{i = 1}^4p(S_i)log2(p(S_i)) = 1.96716 \\    
    H(X^4) &= H(X^1) - 3 H(X | X^\infty) = 5.84397 \\
    \end{aligned}
\end{equation*}
\bibliographystyle{plain}

\newpage
\section*{Zadatak 4}


\begin{center}
\begin{tikzpicture}[scale=0.2]
\tikzstyle{every node}+=[inner sep=0pt]
\draw [black] (56.1,-10.3) circle (3);
\draw (56.1,-10.3) node {$S_1_1$};
\draw [black] (15.5,-41.5) circle (3);
\draw (15.5,-41.5) node {$S_0_0$};
\draw [black] (15.5,-10.3) circle (3);
\draw (15.5,-10.3) node {$S_1_0$};
\draw [black] (56.1,-41.5) circle (3);
\draw (56.1,-41.5) node {$S_0_1$};
\draw [black] (53.1,-10.3) -- (18.5,-10.3);
\fill [black] (18.5,-10.3) -- (19.3,-10.8) -- (19.3,-9.8);
\draw (35.8,-9.8) node [above] {$0/0.2$};
\draw [black] (18.23,-11.544) arc (64.68934:40.22806:107.054);
\fill [black] (54.19,-39.18) -- (54.06,-38.25) -- (53.3,-38.89);
\draw (40.73,-22.94) node [above] {$1\mbox{ }/\mbox{ }0.7$};
\draw [black] (56.1,-38.5) -- (56.1,-13.3);
\fill [black] (56.1,-13.3) -- (55.6,-14.1) -- (56.6,-14.1);
\draw (56.6,-25.9) node [right] {$1\mbox{ }/\mbox{ }0.4$};
\draw [black] (15.5,-13.3) -- (15.5,-38.5);
\fill [black] (15.5,-38.5) -- (16,-37.7) -- (15,-37.7);
\draw (16,-25.9) node [right] {$0\mbox{ }/\mbox{ }0.3$};
\draw [black] (53.408,-40.176) arc (-116.89669:-138.18592:122.677);
\fill [black] (17.47,-12.56) -- (17.63,-13.49) -- (18.38,-12.82);
\draw (31.12,-28.54) node [below] {$0\mbox{ }/\mbox{ }0.6$};
\draw [black] (18.5,-41.5) -- (53.1,-41.5);
\fill [black] (53.1,-41.5) -- (52.3,-41) -- (52.3,-42);
\draw (35.8,-41) node [above] {$1\mbox{ }/\mbox{ }0.8$};
\draw [black] (13.474,-43.696) arc (-14.96249:-302.96249:2.25);
\draw (8.45,-44.77) node [left] {$0\mbox{ }/\mbox{ }0.2$};
\fill [black] (12.52,-41.23) -- (11.88,-40.54) -- (11.62,-41.5);
\draw [black] (57.519,-7.67) arc (179.39312:-108.60688:2.25);
\draw (62.19,-5.02) node [right] {$1\mbox{ }/\mbox{ }0.8$};
\fill [black] (59.05,-9.83) -- (59.85,-10.33) -- (59.86,-9.33);
\end{tikzpicture}
\end{center}

Analogno prethodnom zadatku, dobijamo sistem
\begin{equation*}
    \begin{aligned}
        0 &= -0.8 p(S_{00}) + 0 p(S_{00}) + 0.3 p(S_{00}) + 0 p(S_{00}) \\
0 &= 0.8 p(S_{01}) + -1 p(S_{01}) + 0.7 p(S_{01}) + 0 p(S_{01}) \\
0 &= 0 p(S_{10}) + 0.6 p(S_{10}) + -1 p(S_{10}) + 0.2 p(S_{10}) \\
1 &= 1 p(S_{11}) + 1 p(S_{11}) + 1 p(S_{11}) + 1 p(S_{11}) \\
    \end{aligned}
\end{equation*}
čije rješenje koristimo za ostatak zadatka
\begin{equation*}
    \begin{aligned}
p(S_{00}) &= 0.0857143 \\
p(S_{01}) &= 0.228571 \\
p(S_{10}) &= 0.228571 \\
p(S_{11}) &= 0.457143 \\
H(S_{00}) &= 0.721928\\
H(S_{01}) &= 0.970951\\
H(S_{10}) &= 0.881291\\
H(S_{11}) &= 0.721928\\
H(X\mid X^\infty) &= 0.815273 \\
H(X^2) &= 1.79343 \\
H(X^7) &=  6.68507 \\
    \end{aligned}
\end{equation*}
Rendundansa izvora je
\[
\Re = \frac{2log_2(2) - H(X \mid X^\infty)}{2log_2(2)} = 0.592363
\] a vjerovatnoća pojavljivanja stringa 1110110 je
\[
p(1110110) = p(S_{11}) \cdot p(1 \mid S_{11}) \cdot p(0 \mid S_{11}) \cdot p (1 \mid S_{10}) \cdot p (1 \mid S_{01}) \cdot p (0 \mid S_{11}) \approx 0.4571 \cdot 0.8 \cdot 0.2 \cdot 0.7 \cdot 0.6 \cdot 0.2 \approx 0.006144001
\]

\newpage

\section*{Zadatak 5}




\end{document}
